
\section{Problem statement}

Although the thesis and implemented prototypes will focus on extracting a surface mesh from Enlight's model, the developed algorithms will be as generic as possible and also suitable for similar scenarios. Tukora provides a good summary of representations used in virtual machining in the introductory chapter of this PhD dissertation \cite{virtual_machining_review}.
The following questions should be answered by the proposed thesis:

\begin{itemize}
	\item What is the state of the art in surface reconstruction from implicitly defined geometries similar to Enlight's model (triangle based, subtraction of volumes, non-water-tight volumes)? Can these existing algorithms be categorized to identify common ideas, key concepts and restrictions? Are the found algorithms generic enough to be suitable for different kinds of implicit models (models apart from Enlight, parametric, CSG, functional, etc.)?
	
	\item After at least a prototypic implementation of selected algorithms, which algorithms excel in runtime, memory requirements, asymptotic complexity, visual quality, generated errors, divergence from exact solution, numerical stability, mesh quality (manifold, orientable, no boundary edges, Delaunay\footnote{A triangulation is called Delaunay when the circumcircle of each triangle does not contain a vertex of another triangle. Delaunay triangulations produce very regular and visually appealing triangle meshes.}), feature preservation, adaptivity in triangle size/count, etc.? Can these algorithms be parallelized and potentially (maybe with restrictions) run in real-time?
	
	\item After intensive testing on selected models, can a "best" algorithm be identified? Are there cases in which some algorithms perform better than others and vice-versa (cf. no free lunch theorem \footnote{The no free lunch theorem by Wolpert and Macready states that for any given optimization algorithm there will always be an optimization problem where this algorithm is outperformed by another algorithm \cite{no_free_lunch}. })? What are the criteria that have to be satisfied for an algorithm to run "well"? Can a mechanism be developed to select the algorithm best suited for a given implicit model?
\end{itemize}
