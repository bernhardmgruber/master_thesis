
 
\section{Motivation}

Many applications in different fields have to deal with the representation of geometries, their volumes and surfaces. As visualization often plays an important role, being able to efficiently render a piece of geometry becomes a decisive factor when choosing an appropriate data structure. Modern hardware architectures and well-established graphic APIs like OpenGL and DirectX are commonly optimized towards rasterization and require explicit surface meshes (\ie triangles). In addition to rendering, triangle meshes are furthermore well established means to process, store, edit, distribute and sell geometric models resulting in broad support in many software products.

However, despite their great suitability and acceptance triangle based representations also have their drawbacks and a numerous amount of problems can be addressed better using different, often implicit methods. In many of these situations efficient rendering is only a secondary requirement. Implicit models like functional and parametric surfaces shine in mathematical exactness, expressiveness and memory requirements. A sphere for example is easily expressed as an equation with radius and center as parameters, describing an exact surface. Triangle meshes are in most cases only approximations of such shapes and require an appropriate resolution (\ie triangle count) to achieve the desired visual quality. Exactness is vital in fields like CAD (computer-aided design) and CAM (computer-aided manufacturing).
Some kinds of problems may also benefit greatly in runtime when solved for implicit models. Questions like whether a point is inside a volume or if two volumes intersect can be more easily answered on a few mathematically defined shapes than on a large set of triangles. Applications range from ray tracing photo realistic images to collision detection in physics engines or milling machines. 
Sometimes an implicit representation is easier to create. An example would be CSG where a complex shape is constructed using set operations like union, intersection or subtraction on simple primitives. The resulting model is described using a tree where each node is an operator and each leaf a primitive. CSG is supported by a wide range of modelling tools and CAD kernels.
Finally, some problems are just inherently implicit. Simulating material removal processes in CAM for example is often described as the subtraction of swept volumes (volumes swept by the cutting tool which are themselves described as the cutter's geometry moved along a specified path) from an initial volume. The final work piece can be seen as subtraction of all swept volumes from the initial volume.

Explicit and implicit representations both have their specific usage scenarios. Nevertheless, it is sometimes necessary to convert between them.
A common scenario in CAD for example is to export an implicitly described model (\eg CSG tree) as an explicit triangle mesh. This process is called tessellation or triangulation. The quality of the triangle approximation often depends on a parameterized resolution which in most cases directly influences the number of generated triangles. Several algorithms exist trying to create triangles adaptively, using more and small triangles where detail is necessary to preserve a models so-called features.
The inverse process does also exist where algorithms try to recognize shapes in given triangle sets. An application might be to reconstruct shapes from a triangle set outputted by a simulation and measure parameters of the recognized shape. In CAE (computer-aided engineering) for example, after simulating the milling of a cylinder, a generated triangle set may be recognized again as a cylinder and parameters (\eg radius, height) of the recognized shape can be compared with the initial model put in the simulation to verify the correctness of the simulated machining process.

This thesis will focus solely on the conversion of implicit into explicit models.