
\section{State of the art}

After a survey of existing approaches to surface reconstruction at least four different classes of algorithms have been identified.

\begin{description}
	
\item[Direct intersection] \hfill \\
Directly intersecting each triangle of one volume with another is the most direct approach and described for example by Rosen to smooth the sharp intersection line between two meshes \cite{mesh_intersection}. Triangle-triangle intersection is described by M\"oller \cite{tri_tri_intersection_moller} as well as Tropp, Tal and Shimshoni \cite{tri_tri_intersection_2}. Each intersected triangle can than be re-triangulated based on the cut segments from other triangles (constraints). A possible algorithm is CDT (constrained Delaunay triangulation) and was first described by Chew \cite{CDT} and refined by Sloan \cite{CDT_fast}. After re-triangulating intersected triangles, all triangles not belonging to the new surface are removed. Gong demonstrates a very similar approach in his master thesis \cite{cutter_workpiece_engagement}.

\item[Point cloud based] \hfill \\
Point clouds are data structures where an object as approximated by a set of points with optional normal vectors. These points can be obtained \eg by sampling an implicit model. These point clouds can then be triangulated using different algorithms. 
Amenta, Bern and Kamvysselis present an algorithm for reconstructing Delaunay triangulations from point clouds with certain quality guarantees for "good" inputs \cite{vornoi_surface_reconstruction}.
Hoppe \etal present an algorithm which uses a signed distance function to describe the distance from each point to the estimated surface. The contour of this function is then traced by a marching cubes variant to extract an iso surface (triangle mesh) \cite{surface_reconstruction}.
Bernardini \etal describe the BPA (ball-pivoting algorithm) for triangulating a point cloud which also contains points inside the cloud which are not relevant for the surface \cite{BPA}.


\item[Voxel based] \hfill \\

As Enlight already uses a regular grid data structure to organize triangles and classify the grid's voxel as inside, outside and surface, algorithms can be used which can directly operate on these cells/voxels.
Kobbelt \etal propose a post-processing step to the commonly known marching cubes algorithm to extract/preserve features of the original surface \cite{extended_marching_cubes}. This algorithm is sometimes also referred to as extended marching cubes and makes use of triangles, points, implicit surface functions and scalar distance fields inside each voxel.


\item[Dexel based] \hfill \\

A dexel (depth pixel) is a list of intersections of a ray with an geometric model ordered by depth. A matrix of such dexels is called a ray-rep (ray representation) and generated using a ray cast of parallel rays through a geometric model. Dexel based representation of objects have advantages when simulating material removal processes. To increase the quality of the representation in cases where the geometry becomes parallel to the rays used to create the dexels, three dexel matrices are used, each created along each coordinate axis. This representation is called triple ray representation and was first described by Benouamer and Michelucci \cite{tridexel_intersection} as a method of combining CSG trees and B-reps. A feature conserving algorithm for converting such representations (referred to as tri-dexel volumetric models in the corresponding paper) into polygon meshes is demonstrated by Ren, Zhu and Lee \cite{tridexel_reconstruction}.

\end{description}
