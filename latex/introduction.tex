\chapter{Introduction}
\label{ch:introduction}

\section{Motivation and background}
\label{sec:motivation}

Many scientific fields have to deal with the representation of geometries, their volumes and surfaces.
As visualization is often vital, being able to efficiently render geometries becomes a decisive factor when choosing an appropriate data structure.
Modern hardware architectures and well-established graphic APIs like OpenGL and DirectX are commonly optimized towards rasterization and require explicit, polygonal surface meshes, \eg triangle meshes.
Furthermore, many scientific applications in fields like structural engineering, material science or fluid dynamics are based on finite element methods which rely on triangle meshes to approximate their underlying mathematical models.
Additionally, triangle meshes are a well-established way to process, store, edit, distribute and sell geometric models resulting in broad support in many software products.


However, despite their great suitability and acceptance, triangle-based representations also have their drawbacks and a numerous amount of problems can be addressed better using different representations. %(\eg CSG trees, dexel images or functional representations).
In many of these situations efficient rendering is only a secondary requirement.
Functional models like parametric surfaces or surfaces given implicitly by equations excel in mathematical exactness, expressiveness and memory requirements.
A sphere for example is easily expressed as an equation with radius and center as parameters, describing an exact surface.
Triangle meshes are in most cases only approximations to such shapes and require an appropriate resolution, \ie triangle count, to achieve the desired visual quality.
%
Some problems can also be solved much faster and simpler using representations other than triangle meshes.
Testing if a point is inside a volume or if two volumes intersect is more easily done on a few mathematically defined shapes than on a large set of triangles.
Applications range from ray tracing photo realistic images to collision detection in physics engines or milling machines.
Boolean set operations are much easier calculated on representations like z-maps and dexel images.
These are heavily used in virtual machining to simulate material removal, \eg in milling, or addition, \eg in 3-dimensional printing.
%
Sometimes representations other than triangle meshes are easier to create.
An example would be constructive solid geometry (CSG) where a complex shape is constructed using set operations like union, intersection or subtraction on simple primitives, e.g. cubes, cylinders or spheres.
The resulting model is described using a tree where each node is an operator and each leaf a primitive.
CSG is supported by a wide range of modeling tools and computer-aided design (CAD) kernels.
Another example are dexel images when used for material removal in sculpturing.
%
Finally, some surface descriptions are just inherently non-explicit, \eg described procedurally.
Simulating milling processes in computer-aided manufacturing (CAM) for example is usually described as the subtraction of swept volumes from an initial volume.
Swept volumes are volumes swept by a cutting tool. They are therefore described as the cutter's geometry moved along a specified path.
The final workpiece is obtained by subtraction of all swept volumes from the initial volume.


Despite polygon-based and alternative representations for surfaces both having their specific usage scenarios, it is sometimes necessary to transform one into the other.
A common scenario in CAD for example is to export a model described using a CSG tree or dexel image as a triangle mesh.
The quality of the triangle approximation usually depends on a user-defined resolution which in many cases directly influences the number of generated triangles.
Several algorithms exist trying to create meshes adaptively, using more and small triangles only where detail is necessary to preserve a models features.
%
The inverse process does also exist where algorithms try to recognize shapes in given triangle sets.
An example is found in computer-aided engineering (CAE), specifically CAM verification: After simulating the milling of a cylinder, a generated triangle set may be recognized again as a cylinder. The parameters of this recognized shape, \eg radius and height, can be compared with the initial CAD model put in the simulation to verify the correctness of the simulated machining process.


\section{Previous work}

Due to the broad field of surface representations and the author's previous experience, this thesis narrows its focus on models used in subtractive manufacturing simulations and focuses solely on the transformation of these models into explicit triangle representations.
During the authors work at the RISC Software GmbH a visualization and solid modeling software was developed during two research projects, Enlight and Engrave (\cf previous work in \cref{ch:previous_work}), to visualize and simulate subtractive manufacturing as done by milling machines.
Development is continued under the name virtual machining library (VML) and includes a feature for surface reconstruction/extraction from the internal data model, which forms the practical foundation of this thesis.
Although the presented implementations are tailored to this application context, the underlying algorithms are discussed as general as possible to explore broader usage scenarios.


\section{Problem statement}
\label{sec:problem}

The focus of the implementation underlying this thesis is to find, evaluate and prototypically implement multiple strategies to extract a triangulated surface mesh from the data model used inside the VML.
The thesis on top of these prototypes provides a comprehensive documentation, analysis and discussion of the implemented algorithms and strategies.
In detail, the following questions are answered:

\begin{enumerate}
	\item What is the state of the art in surface reconstruction from geometric data models similar to the VML's?
	What are common ideas, key concepts and restrictions of these existing algorithms and how can they be categorized?
	Are these algorithms generic enough to be suitable for other kinds of models, \ie models apart from the VML?

	\item After a prototypic implementation of selected algorithms, which excel in
	\begin{itemize}
		\item runtime,
		\item memory requirements,
		\item asymptotic complexity,
		\item visual quality,
		\item generated errors,
		\item divergence from exact solution,
		\item numerical stability,
		\item mesh quality, \eg manifold, orientable, no boundary edges, Delaunay conforming,
		\item feature preservation,
		\item adaptivity in triangle size/count,
		\item \etc?
	\end{itemize}
	Are these algorithms suitable for parallelization and how good do they scale?

	\item After intensive testing on selected models, is it possible to identify a \enquote{best} algorithm?
	Are there cases in which some algorithms perform better than others and vice-versa, \cf no free lunch theorem in optimization\footnote{The no free lunch theorem by Wolpert and Macready states that for any given optimization algorithm there will always be an optimization problem where this algorithm is outperformed by another one \cite{no_free_lunch}.}?
	What are the criteria that have to be satisfied for an algorithm to run \enquote{well}?
\end{enumerate}


\section{Goal}
\label{sec:goal}

The goal of this thesis is to provide a state of the art overview of algorithms and methods used to reconstruct explicit triangle meshes from different representations such as the one used inside the VML.
In reference to the first problem statement, a categorization of the presented algorithms is shown to group common concepts and compare the algorithms based on their approaches, area of application, supported data structures and restrictions.

After this overview, the thesis discusses the prototypic implementation of selected algorithms.
These implementations are based on the data model used inside the VML.
Nevertheless, adaptations to support different kinds of data structures are pointed out where possible.
All algorithms are compared using the aspects given in the second problem statement.
Although time constrained surface extracting is not focused, estimating the potential of each algorithm to be used in real-time scenarios has been tried, as this is an important scenario in collision detection and avoidance during machining.
Furthermore, as modern hardware architectures become increasingly parallel and heterogeneous, this thesis provides hints and estimates about the suitability to parallelize the chosen algorithms.

Additionally, a suite of representative test models has been created and used to benchmark the prototyped approaches.
The primary goal thereby is to point out the strengths and limitations of the implemented prototypes on various difficulties of the provided test scenes, \eg feature detection and errors.
Finally, estimates and advices are given about which algorithms and strategies work best for which kind of input.

This thesis does not provide a detailed introduction into virtual machining, ray tracing nor computer graphics.
Furthermore, all implementations of algorithms are prototypes and may not be suitable for all kinds of inputs nor for direct use in production.
The presented algorithms are not optimized in terms of performance.
However, the potential for optimizations is discussed.


\section{Chapter overview}
\label{sec:chapter_overview}

In \cref{ch:fundamentals} a coverage of the most important terms related to solid modeling, meshes and virtual machining as well as a comprehensive list of wide-spread representations used in CAD and CAM are discussed.

\Cref{ch:previous_work} contains background information on the VML, the software this thesis is based on.
After a bit of history on this project, the VML's data model and a few core algorithms are elaborated to allow a deeper understanding of various design aspects of the presented implementations. These include the VML's regular grid data structure, a triangle elimination strategy called classification and the ray casting subsystem used for visualization.

\Cref{ch:state_of_the_art} provides a literature review of existing algorithms and approaches which could be, fully or partially, used to tackle the problem of surface reconstruction for the VML's data model.
The presented references include means for direct triangle intersection, surface reconstruction from point clouds and triangulation of dexel-based as well as voxel-based representations.

\Cref{ch:direct_intersection} discusses the first implemented approach to reconstruct a surface from the VML's data model.
It relies on directly intersecting the triangles of the swept volumes stored within the VML.
This method is the most naive, but, in theory, most exact.

\Cref{ch:tri_dexel} describes the second implementation which is based on a tri-dexel representation.
Utilizing an axis-aligned raycast to sample the VML's data model, a tri-dexel image is created.
This sampled data model is regularized to fight numeric issues and sampling errors.
A subsequent feature reconstruction pass greatly enhances the visual quality of the output.
At the end, a novel extension for adaptive resampling is introduced.

\Cref{ch:point_cloud_based} builds on the idea of the axis-aligned raycast to sample a point cloud of the VML's data model.
As surface reconstruction algorithms for point clouds are greatly available in literature, the potential of these kinds of algorithms is discussed.
The main example is a variant of the Ball Pivoting Algorithm used by the VML for swept volume computation.
Furthermore, a Poisson surface reconstruction algorithm is also shortly presented.

%\dummytext{2}
