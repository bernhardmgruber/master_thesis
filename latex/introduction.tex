\chapter{Introduction}
\label{ch:introduction}

\section{Motivation and background}
\label{sec:motivation}

Many applications in different fields of science have to deal with the representation of geometries, their volumes and surfaces.
As visualization often plays an important role, being able to efficiently render a piece of geometry becomes a decisive factor when choosing an appropriate data structure.
Modern hardware architectures and well-established graphic APIs like OpenGL and DirectX are commonly optimized towards rasterization and require explicit, polygonal surface meshes (\eg triangle meshes).
Furthermore, many scientific applications in fields like structural engineering, material science or fluid dynamics are based on finite element methods with rely on triangle meshes to approximate their underlying mathematical models.
Additionally, triangle meshes are well established means to process, store, edit, distribute and sell geometric models resulting in broad support in many software products.


However, despite their great suitability and acceptance, triangle-based representations also have their drawbacks and a numerous amount of problems can be addressed better using different representations. %(\eg CSG trees, dexel images or functional representations).
In many of these situations efficient rendering is only a secondary requirement.
Implicit models like functional and parametric surfaces shine in mathematical exactness, expressiveness and memory requirements.
A sphere for example is easily expressed as an equation with radius and center as parameters, describing an exact surface.
Triangle meshes are in most cases only approximations of such shapes and require an appropriate resolution (\ie triangle count) to achieve the desired visual quality.
%
Some kinds of problems may also benefit greatly in runtime when solved for implicit models.
Questions like whether a point is inside a volume or if two volumes intersect can be more easily answered on a few mathematically defined shapes than on a large set of triangles.
Applications range from ray tracing photo realistic images to collision detection in physics engines or milling machines.
Boolean set operations are much easier calculated on representations like z-maps and dexel images.
These are heavily used in virtual machining to simulate material removal (\eg milling) or addition (\eg 3-dimensional printing).
%
Sometimes representations other than triangle meshes are easier to create.
An example would be CSG (constructive solid geometry) where a complex shape is constructed using set operations like union, intersection or subtraction on simple primitives (e.g. cubes, cylinders or spheres).
The resulting model is described using a tree where each node is an operator and each leaf a primitive.
CSG is supported by a wide range of modeling tools and CAD (computer-aided design) kernels.
Another example are dexel images when used for material removal in sculpturing.
%
Finally, some surface descriptions are just inherently non-explicit (\eg described procedurally).
Simulating milling processes in CAM (computer-aided manufacturing) for example is often described as the subtraction of swept volumes from an initial volume.
Swept volumes are volumes swept by a cutting tool which are themselves described as the cutter's geometry moved along a specified path.
The final workpiece can be seen as subtraction of all swept volumes from the initial volume.


Despite polygon-based and alternative representations for surfaces both having their specific usage scenarios, it is sometimes necessary to transform one into the other.
A common scenario in CAD for example is to export an model described using a CSG tree or dexel image as a triangle mesh.
The quality of the triangle approximation usually depends on a parameterized resolution which in many cases directly influences the number of generated triangles.
Several algorithms exist trying to create meshes adaptively, using more and small triangles only where detail is necessary to preserve a models features.
%
The inverse process does also exist where algorithms try to recognize shapes in given triangle sets.
An application might be to reconstruct shapes from a triangle set outputted by a simulation and measure parameters of the recognized shape.
In CAE (computer-aided engineering) for example, after simulating the milling of a cylinder, a generated triangle set may be recognized again as a cylinder and parameters of the recognized shape (\eg radius, height) can be compared with the initial model put in the simulation to verify the correctness of the simulated machining process.


\section{Previous work}

Due to the broad field of surface representations and the authors previous experience, this thesis will narrow its focus on models used in subtractive manufacturing simulations and focus solely on the transformation of these models into explicit triangle representations (triangle surface reconstruction).
During the authors work at the RISC Software GmbH a visualization and solid modeling software has been developed during two research projects, Enlight and Engrave (\cf previous work in chapter \ref{ch:previous_work}), to visualize and simulate subtractive manufacturing as done by milling machines.
Development is continued under the name VML (Virtual Machining Library) and includes a feature for surface reconstruction/extraction from the internal data model, which forms the practical foundation of this thesis.
Although the presented implementations are tailored to this application context, the underlying algorithms are discussed as general as possible to explore broader usage scenarios.


\section{Problem statement}
\label{sec:problem}

The focus of the implementations underlying this thesis is to find, evaluate and prototypically implement multiple strategies to extract a triangulated surface mesh from the data model used inside the VML.
On top of these prototypes this thesis provides a comprehensive documentation, analysis and discussion of the implemented algorithms and strategies.
In detail, the following questions will be answered:

\begin{enumerate}
	\item What is the state of the art in surface reconstruction from geometric data models similar to the VML's?
	Can these existing algorithms be categorized to identify common ideas, key concepts and restrictions?
	Are the found algorithms generic enough to be suitable for other kinds of models (models apart from the VML, \cf)?
	
	\item After at least a prototypic implementation of selected algorithms, which algorithms excel in runtime, memory requirements, asymptotic complexity, visual quality, generated errors, divergence from exact solution, numerical stability, mesh quality (manifold, orientable, no boundary edges, Delaunay), feature preservation, adaptivity in triangle size/count, \etc?
	How well can these algorithms be parallelized and how good do they scale?
	
	\item After intensive testing on selected models, can a \enquote{best} algorithm be identified?
	Are there cases in which some algorithms perform better than others and vice-versa (\cf no free lunch theorem in optimization\footnote{The no free lunch theorem by Wolpert and Macready states that for any given optimization algorithm there will always be an optimization problem where this algorithm is outperformed by another one \cite{no_free_lunch}.})?
	What are the criteria that have to be satisfied for an algorithm to run \enquote{well}?
\end{enumerate}


\section{Goal}
\label{sec:goal}

The goal of this thesis is to provide the reader with a state of the art overview of algorithms and methods used to reconstruct explicit triangle meshes from different representations such as the one used inside the VML.
In reference to the first problem statement, a categorization of the presented algorithms will be shown to group common concepts and compare the algorithms based on their approaches, area of application, supported data structures and restrictions.

After this overview, the thesis will discuss the prototypic implementation of selected algorithms.
These implementations will be based on the data model used inside the VML.
Nevertheless, the author will point out how certain implementations may be adapted for different kind of data structures where possible.
All algorithms will be compared using the aspects given in the second problem statement.
Although time constrained surface extracting will not be focused, estimating each algorithms potential to be used in real-time scenarios will be tried, as this is an important sector of collision detection and avoidance during machining.
Furthermore, as modern hardware architectures become increasingly parallel and heterogeneous, this thesis provides hints and estimates about the suitability to parallelize the chosen algorithms.

Additionally, a suite of representative test models has been created and used to benchmark the prototyped approaches.
The primary goal thereby is to point out the strengths and limitations of the implemented prototypes and to compare them again on their success on various difficulties of the provided test scenes (\eg feature detection and errors).
Finally, this thesis will be able to give estimates and advices about which algorithms and strategies work best for which kind of input.

This thesis will not provide a detailed introduction into virtual machining, ray tracing nor computer graphics.
Furthermore, all implementations of algorithms will be prototypes and may not be suitable for every kind of input nor for direct use in production.
The presented algorithm are also not tuned for performance.
However, potential for optimizations will be discussed.


\section{Chapter overview}
\label{sec:chapter_overview}

After this introduction, chapter \ref{ch:fundamentals} will continue with a coverage of the most important terms related to solid modeling, meshes and virtual machining as well as a comprehensive list of wide-spread representations used in CAD and CAM.

Chapter \ref{ch:previous_work} will give background information on the VML on which the implementation of this thesis is based on.
After a bit of history on this project, the VML's data model and a few core algorithms are elaborated to allow the reader a deeper understanding of various design aspects of the presented implementations. These include the VML's regular grid data structure, a triangle elimination strategy called classification and the ray casting subsystem used for visualization.

Chapter \ref{ch:state_of_the_art} continues with literature review of existing algorithms and approaches which could be (partially) used to tackle the problem of surface reconstruction for the VML's data model.
The presented references include means for direct triangle intersection, surface reconstruction from point clouds, triangulation of dexel based as well as voxel based representations.

\dummytext{2}