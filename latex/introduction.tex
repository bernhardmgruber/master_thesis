\chapter{Introduction}

\section{Background and motivation}

Many applications in different fields of science have to deal with the representation of geometries, their volumes and surfaces.
As visualization often plays an important role, being able to efficiently render a piece of geometry becomes a decisive factor when choosing an appropriate data structure.
Modern hardware architectures and well-established graphic APIs like OpenGL and DirectX are commonly optimized towards rasterization and require explicit surface meshes (\ie triangles).
Furthermore, many scientific applications in fields like structural engineering, material science or fluid dynamics are based on finite element methods with rely on triangle meshes to approximate their underlying mathematical models.
Additionally, triangle meshes are well established means to process, store, edit, distribute and sell geometric models resulting in broad support in many software products.


However, despite their great suitability and acceptance, triangle-based representations also have their drawbacks and a numerous amount of problems can be addressed better using different, often implicit, methods (\cf examples by Menon \cite{implicit_techniques}). %TODO write some examples down
In many of these situations efficient rendering is only a secondary requirement.
Implicit models like functional and parametric surfaces shine in mathematical exactness, expressiveness and memory requirements.
A sphere for example is easily expressed as an equation with radius and center as parameters, describing an exact surface.
Triangle meshes are in most cases only approximations of such shapes and require an appropriate resolution (\ie triangle count) to achieve the desired visual quality.
Exactness is vital in fields like CAD (computer-aided design) and CAM (computer-aided manufacturing).
%
Some kinds of problems may also benefit greatly in runtime when solved for implicit models.
Questions like whether a point is inside a volume or if two volumes intersect can be more easily answered on a few mathematically defined shapes than on a large set of triangles.
Applications range from ray tracing photo realistic images to collision detection in physics engines or milling machines.
%
Sometimes an implicit representation is easier to create.
An example would be CSG (constructive solid geometry) where a complex shape is constructed using set operations like union, intersection or subtraction on simple primitives (e.g. cubes, cylinders or spheres).
The resulting model is described using a tree where each node is an operator and each leaf a primitive.
CSG is supported by a wide range of modelling tools and CAD kernels.
%
Finally, some problems are just inherently implicit.
Simulating material removal processes in CAM for example is often described as the subtraction of swept volumes (volumes swept by a cutting tool which are themselves described as the cutter's geometry moved along a specified path) from an initial volume.
The final work piece can be seen as subtraction of all swept volumes from the initial volume.


Despite explicit and implicit representations for surfaces both having their specific usage scenarios, it is sometimes necessary to transform one into the other.
A common scenario in CAD for example is to export an implicitly described model (\eg CSG tree) as an explicit triangle mesh.
This process is called tessellation or triangulation.
The quality of the triangle approximation usually depends on a parameterized resolution which in many cases directly influences the number of generated triangles.
Several algorithms exist trying to create meshes adaptively, using more and small triangles only where detail is necessary to preserve a models features.
%
The inverse process does also exist where algorithms try to recognize shapes in given triangle sets.
An application might be to reconstruct shapes from a triangle set outputted by a simulation and measure parameters of the recognized shape.
In CAE (computer-aided engineering) for example, after simulating the milling of a cylinder, a generated triangle set may be recognized again as a cylinder and parameters of the recognized shape (\eg radius, height) can be compared with the initial model put in the simulation to verify the correctness of the simulated machining process.


Due to the broad field of surface representations and the authors previous experiences, this thesis will narrow its focus to models used in subtractive manufacturing simulations and focus solely on the transformation of these models into explicit triangle representations.
During the authors work at the RISC Software GmbH a visualization and solid modelling software has been developed during two research projects, Enlight and Engrave, to visualize and simulate subtractive manufacturing as done by milling machines.
Development is continued under the name VML (Virtual Machining Library) and includes a feature for surface reconstruction/extraction from the internal data model, which forms the practical foundation of this thesis.
Although the presented implementations are tailored to this application context, the underlying algorithms are discussed as general as possible to explore broader usage scenarios.


\section{Problem statement}


Although the thesis and implemented prototypes will focus on extracting a surface mesh from Enlight's model, the developed algorithms will be generic and also suitable for similar scenarios.
Tukora provides a good summary of representations used in virtual machining in the introductory chapter of this PhD dissertation \cite{virtual_machining_review}.
The following questions will be answered:

\begin{enumerate}
	\item What is the state of the art in surface reconstruction from implicitly defined geometries similar to Enlight's model (triangle-based, subtraction of volumes, non-water-tight volumes)? Can these existing algorithms be categorized to identify common ideas, key concepts and restrictions? Are the found algorithms generic enough to be suitable for different kinds of implicit models (models apart from Enlight, parametric, CSG, functional, \etc)?
	
	\item After at least a prototypic implementation of selected algorithms, which algorithms excel in runtime, memory requirements, asymptotic complexity, visual quality, generated errors, divergence from exact solution, numerical stability, mesh quality (manifold, orientable, no boundary edges, Delaunay\footnote{A triangulation is called Delaunay when the circumcircle of each triangle does not contain a vertex of another triangle. Delaunay triangulations produce very regular and visually appealing triangle meshes.}), feature preservation, adaptivity in triangle size/count, \etc? Can these algorithms be parallelized and potentially (maybe with restrictions) run in real-time?
	
	\item After intensive testing on selected models, can a \enquote{best} algorithm be identified? Are there cases in which some algorithms perform better than others and vice-versa (\cf no free lunch theorem in optimization\footnote{The no free lunch theorem by Wolpert and Macready states that for any given optimization algorithm there will always be an optimization problem where this algorithm is outperformed by another one \cite{no_free_lunch}. })? What are the criteria that have to be satisfied for an algorithm to run \enquote{well}? Can a mechanism be developed to select the algorithm best suited for a given implicit model?
\end{enumerate}


Transform implicit models (similar to Enlight) into triangle mesh

\section{Goal}
Research, prototype and compare different approaches on various test scenes
