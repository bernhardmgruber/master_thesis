\documentclass[
10pt,
a4paper,
draft,
%numbers=noenddot
]
{scrreprt}

% ====================================================================================================
% ===   Packages                                                                                   ===
% ====================================================================================================
\usepackage{scrhack}
\usepackage[utf8]{inputenc}
%\usepackage[T1]{fontenc}
\usepackage{amsmath}
\usepackage{amsfonts}
\usepackage{amssymb}
\usepackage{mathtools}
\usepackage[binary-units=true,group-digits=false]{siunitx}
\usepackage[final]{graphicx}
\usepackage{subcaption}
\usepackage[style=numeric,backend=biber,urldate=iso8601]{biblatex}
%\usepackage{listings}
\usepackage{xspace}
\usepackage[iso]{datetime}
\usepackage{tabularx}
\usepackage{multirow}
\usepackage{booktabs}
\usepackage[final]{pdfpages}
\usepackage{lipsum}
\usepackage{color}
\usepackage{chngcntr}
\usepackage{fixltx2e}
\usepackage{float}
\usepackage{hyperref}
\usepackage[plain]{algorithm}
\usepackage[noend]{algpseudocode}


% ====================================================================================================
% ===   Packages config                                                                            ===
% ====================================================================================================

%
% biblatex
%
\addbibresource{references.bib}
\let\cite\parencite

%
% graphics
%
\graphicspath{ {./images/} }

%
% listings
%
%\lstdefinelanguage{CPP}[ISO]{C++}
%{
%	morekeywords={size_t},
%	morekeywords={inline}
%}

%\lstset
%{
%	numberbychapter=false,
%	tabsize=2,
%	captionpos=b,
%	basicstyle=\ttfamily{},
%	aboveskip=\parskip,
%% keywordstyle=\color{blue}\bfseries,
%	commentstyle=\color{gray},
%% stringstyle=\color{red},
%	showstringspaces=false,
%	breaklines=true,
%	numbers=left,
%	morekeywords={size_t}
%%	frame=single,
%% backgroundcolor=\color{lightgray}
%}

%
% algorithm
%
\newcommand{\var}[1]{\ensuremath{\mathit{#1}}}
\MakeRobust{\Call} % to allow nesting \Call, see: http://tex.stackexchange.com/questions/16046/how-to-nest-call-in-algorithmicx

\algnewcommand\algorithmicto{\textbf{ to }}
\algnewcommand\To{\algorithmicto}

\algnewcommand\algorithmictrue{\textbf{true}\xspace}
\algnewcommand\True{\algorithmictrue}

\algnewcommand\algorithmicfalse{\textbf{false}\xspace}
\algnewcommand\False{\algorithmicfalse}

% from: http://tex.stackexchange.com/questions/74880/algorithmicx-package-comments-on-a-single-line
\algnewcommand{\LineComment}[1]{\State \(\triangleright\) #1}

\newcommand{\out}{\mathord{\uparrow}}

% from: http://tex.stackexchange.com/questions/80140/line-numbering-in-algorithmic
\makeatletter
\newcommand{\setalglineno}[1]{%
	\setcounter{ALG@line}{\numexpr#1-1}}
\makeatother

%
% chngcntr
%
\counterwithout{figure}{chapter}
\counterwithout{table}{chapter}

%
% siunitx
%
\DeclareSIUnit{\nothing}{\relax}

%
% misc
%
% Increase line spacing for wbackfrieder for corrections
%\linespread{1.6}

% ====================================================================================================
% ===   Macros                                                                                     ===
% ====================================================================================================

% e.g. i.e. etc. et al.
\newcommand*{\eg}{e.g.\@\xspace}
\newcommand*{\Eg}{E.g.\@\xspace}
\newcommand*{\ie}{i.e.\@\xspace}
\newcommand*{\cf}{cf.\@\xspace}
\newcommand*{\aka}{aka.\@\xspace}
\makeatletter
\newcommand*{\etc}{%
	\@ifnextchar{.}%
	{etc}%
	{etc.\@\xspace}%
}
\newcommand*{\etal}{%
	\@ifnextchar{.}%
	{et al}%
	{et al.\@\xspace}%
}
\makeatother

\newcommand{\dummytext}[1]{\textcolor{red}{\lipsum[1-#1]}}

\newcommand{\ceil}[1]{\lceil#1\rceil}
\newcommand{\floor}[1]{\lfloor#1\rfloor}

\newcommand{\sub}[1]{\textsubscript{#1}}

\renewcommand{\mod}{\bmod}

\begin{document}
% ====================================================================================================
% ===   Front matter                                                                               ===
% ====================================================================================================

%\title{Detailed and adaptive surface reconstruction of implicitly defined geometries}
\title{Surface reconstruction from models for subtractive manufacturing simulation}
\author{Bernhard Manfred Gruber}
\date{\today}

% Three into pages
%\setcounter{page}{4}

\includepdf{titlepage}
%\maketitle
%\thispagestyle{empty}
\clearpage

%\pagenumbering{Roman}

%\chapter*{Declaration}

I hereby declare and confirm that this thesis is entirely the result of my own original work.
Where other sources of information have been used, they have been indicated as such and properly acknowledged.
I further declare that this or similar work has not been submitted for credit elsewhere.

\vspace{3cm}

\noindent
\parbox{6cm}{
	\centering
	\rule{\linewidth}{1pt}\\
	Date
}
\hfill
\parbox{6cm}{
	\centering
	\rule{\linewidth}{1pt}\\
	Bernhard Manfred Gruber
}
%\chapter*{Acknowledgments}

Alexander Leutgeb, Michael Hava
%\chapter*{Kurzfassung}

\begin{otherlanguage}{ngerman}
	Am Beispiel der Virtual Modeling Library (VML), die an der RISC Software GmbH entwickelt wird, werden in diese Arbeit Algorithmen zur Oberflächenrekonstruktion aus Datenmodellen zur Simulation zerspanender Bearbeitungsprozesse diskutiert.
	Zur Beschreibung eines Werkstücks einer derartigen Bearbeitung, verwaltet die VML Dreiecksmodelle des Ausgangs- und einer Serie an Abzugsvolumen, ähnlich einem CSG Baum.
	Unter Verwendung eines angepassten Raycastingprozesses werden Oberflächenpunkte des Werkstücks abgetastet die eine Visualisierungen der Szene zu ermöglichen.

	Für bestimmte Anwendungsfälle ist jedoch eine explizite Repräsentation des Werkstücks als Oberflächennetz gewünscht.
	Zur Erzeugung eines derartigen Netzes werden drei Methoden vorgestellt.

	Die erste errechnet die Boolschen Subtraktionen der Ausgangs- und Abzugsvolumen durch direkte Verschneidung der jeweiligen Dreiecksmodelle.
	Obwohl für einfache Szenen exakte Resultate berechnet werden können, scheitert diese Methode an komplexen Szenen aufgrund numerischer Probleme.

	Die zweite Methode tastet unter Verwendung eines adaptierten Raycastings ein Tri-Dexel-Bild ab und rekonstruiert daraus ein Dreiecksnetz.
	Mit durchwegs guter Robustheit, einem visuell vertretbaren Resultat und exzellenter Qualität des Netzes, funktioniert diese Methode gut in allen getesteten Szenarios.
	Zusätzlich verbessert eine neuartige Erweiterung des Tri-Dexel Algorithmus die Rekonstruktion von Details auf Kosten kleiner Löcher in der erzeugten Oberfläche.

	Als dritte Methode kann ein Raycast auch zur Abtastung einer Punktwolke der Werkstücksoberfläche herangezogen werden.
	Am Beispiel des Ball Pivoting Algorithm der VML und der Implementierung einer Poisson Oberflächenrekonstruktion in MeshLab werden noch die Verwendung von punktwolkenbasierten Algorithmen diskutiert.
	Beide Varianten liefern visuell gute Ergebnisse, obwohl der BPA bei der Erzeugung manigfaltiger Netze in einigen Tests scheitert und die Poisson Rekonstruktion aufgrund ihrer Robustheit gegen Rauschen Details verliert.

	Schlussendlich, auf Basis der vorgestellten Testszenen, produziert die Tri-Dexel basierte Oberflächenrekonstruktion die visuall besten und robustesten Resultate und wird deshalb Gegenstand weiterer Entwicklung sein.
\end{otherlanguage}

\pagebreak

\chapter*{Abstract}

Using the Virtual Modeling Library (VML) developed at the RISC Software GmbH as an example, surface reconstruction algorithms from data models used to simulate subtractive manufacturing are discussed in this thesis.
To describe a machined workpiece created during this process, the VML stores a stock and a series of swept volume triangle meshes, similar to a CSG tree.
Via a custom raycasting process, surface points of the workpiece are sampled to produce visualizations of the scene.

Sometimes, an explicit surface mesh representation of the workpiece is desired.
For the reconstruction of such a mesh, three approaches are presented.

The first one relies on calculating the Boolean subtraction of the maintained stock and swept volumes by direct mesh intersection.
Although producing exact results for simple scenes, this method fails for complex scenes due to numerical issues.

The second approach samples a Tri-dexel image using an adapted raycast and reconstructs a triangle mesh from the tri-dexel representation.
With overall good robustness, an acceptable visual outcome and excellent mesh qualities, this method works well for all tested scenarios.
Additionally, a novel extension to the tri-dexel algorithm further improves the reconstruction of features at the cost of thin holes in the created surface.

As a third method, a raycast may also be used to sample a point cloud of the surface of the workpiece.
By the example of the Ball Pivoting Algorithm of the VML and a Poisson surface reconstruction implementation of MeshLab, the use of point cloud based surface reconstruction algorithms is discussed as well.
Both variants deliver visually good results, although the BPA sometimes fails to create manifold meshes and the Poisson reconstruction loses features due to its robustness against noise.

Concluding, based on the proposed test scenes, the tri-dexel based surface reconstruction approach produces the visually best and most robust results and will therefore be subject to further development.


\tableofcontents
\clearpage

% ====================================================================================================
% ===   Main matter                                                                                ===
% ====================================================================================================

%\pagenumbering{arabic}

\chapter{Introduction}

\section{Motivation and background}
Explicit vs implicit surface models
conversion


\section{Problem statement}
convert implicit models (similar to Enlight) into triangle mesh

\section{Goal}
Research, prototype and compare different approaches on various test scenes

\chapter{Fundamentals and theoretical background (BASICS)}

\subsection{Terms and definitions}


Implicit models may be composed of polygons, parametric surfaces and functional representations (iso surfaces).
Furthermore, many and also different types of such surfaces may be aggregated, referred to as B-rep (boundary representation), or combined using set operations such as in CSG.


triangulation,
triangle meshes,
brep, CSG
manifold, oriented, closed/water-tight, boundaries
delaunay, voronoi, gabriel simplex

Delaunay

A triangulation is called Delaunay when the circumcircle of each triangle does not contain a vertex of another triangle. Delaunay triangulations produce very regular and visually appealing triangle meshes.


\subsection{Surface reconstruction}
problem analysis




\chapter{Previous work (VML)}
\label{ch:previous_work}

As part of project Enlight a method for modeling and visualizing dynamic geometry models has been developed.
It was conducted for research by the RISC Software GmbH in Hagenberg im M\"uhlkreis, Austria.
The project was funded by the political program Regio 13, which aimed to sustainably improve the contestability of regional companies, economic growth and employment inside of Upper Austria.
Co-funded by the European Union as well as Land Ober\"osterreich the program ran from 2007 to 2013.

The models in Enlight are defined by a set of triangle meshes which are combined using set operations (particularly subtraction).
Applications of such a system are primarily found in simulations of material removal processes, a core discipline of machining, also referred to as subtractive manufacturing.
A vital requirement of such systems is to support a large number of fine triangulated meshes (up to hundreds of millions of triangles) to deliver simulations of appropriate quality.
Therefore, well-conceived space partitioning and triangle elimination strategies have been developed to deal with this amount of input.
As a result the simulated model is described implicitly via subtraction of only subsets of the input triangles (volumes are no longer closed/water-tight).
For visualization, this implicit model is sampled using an adapted ray casting approach \cite{enlight}.
By applying a small set of rules, the surface can be recovered to retrieve an image of the model.
Enlight's data model is described in detail in the author's bachelor thesis \cite{bachelor}.

Simulation and visualization of material removal processes such as in Enlight are vital for verifying the correctness of a machining process.
However, there exist numerous scenarios where an explicit surface model of the final product is required.
Many of these situations occur in CAE such as stress, thermal and safety analyses as well as structural optimizations using finite element methods.
Due to the importance for CAE for assuring quality in modern production processes it is highly desired to be able to retrieve an explicit surface mesh of the implicitly described model of Enlight.



more or less practical background

Enlight, RISC Software GmbH, practical bachelor thesis, internship, Enlight/Engrave papers


\chapter{State of the art}
\label{ch:state_of_the_art}

Algorithms for transforming solid representations as well as surface reconstruction algorithms are used in a variety of fields:
From 3D artist tools for virtual sculpting and triangle exporters of CAD software to surface reconstruction from real world laser scans.
In the context of this thesis, the scope of solid representation transformation and surface reconstruction is limited to the field of virtual machining and triangulated manifold outputs.
After a survey of existing approaches to surface reconstruction from models similar to the VML's at least four different classes of algorithms have been identified:

\begin{description}
	
	\item[Direct intersection] \hfill \\
	Directly intersecting each volume with each other (triangle by triangle) is the most direct and theoretically exact approach of calculating the result of a series of boolean subtractions.
	A good description of triangle mesh intersections is given by Rosen to smooth the sharp intersection line between two meshes \cite{mesh_intersection} for aesthetics in video games.
	Triangle-triangle intersection is described by Möller \cite{tri_tri_intersection_moller} as well as Tropp \etal \cite{tri_tri_intersection_2}.
	Each intersected triangle can be split into polygons based on the cut segments from other triangles and retriangulated using the cutting segments as constraints.
	A possible solution for this retriangulation is the constrained Delaunay triangulation (CDT) and was first described by Chew \cite{cdt} and refined by Sloan \cite{cdt_fast}.
	After retriangulating intersected triangles, all triangles not belonging to the new surface are removed.
	A very similar approach is described by Gong in his master thesis \cite{cutter_workpiece_engagement}.
	
	
	\item[Point cloud based] \hfill \\
	Point clouds are data structures where an object is approximated by a set of points with optional normal vectors.
	These points can be obtained \eg by sampling a solid's surface.
	The VML already uses surface sampling of its data model via its raycast based visualization.
	Point clouds can be triangulated using different algorithms:
	Amenta \etal present the crust and power crust algorithm for reconstructing Delaunay triangulations from point clouds with certain quality guarantees for "good" inputs \cite{crust, power_crust}.
	Amenta \etal further introduce the cocone algorithm \cite{cocone} with is extended by Dey \etal with tight cocone \cite{tight_cocone} and robust cocone \cite{robust_cocone}.
	Hoppe \etal present an algorithm which uses a signed distance function to describe the distance from each point to the estimated surface.
	The contour of this function is then traced by a marching cubes variant to extract an iso surface (triangle mesh) \cite{sdf_surface_reconstruction}.
	Bernardini \etal describe the BPA (ball-pivoting algorithm) for triangulating a point cloud which also contains points inside the cloud which are not relevant for the surface \cite{bpa}.
	Based on the BPA, the G2S algorithm (named after the Gabriel 2-simplex criterion) further improved speed and triangle quality by assuming local surface continuity \cite{g2s}.
	
		
	\item[Dexel based] \hfill \\
	Dexel representations are widely used in virtual machining.
	Although the VML's data model stores triangulated manifolds directly, a dexel representation can be easily created based on the existing raycasting system.
	By casting parallel rays through the VML's regular grid from one side to the other and continuing after a surface intersection, a valid dexel image can be constructed.
	When this process is done along the three axes of the Cartesian coordinate system, a tri-dexel image is obtained.
	A feature conserving algorithm for converting such a representation (referred to as tri-dexel volumetric models in the corresponding paper) into polygon meshes is demonstrated by Ren \etal \cite{tridexel_reconstruction}.
	
	
	\item[Voxel based] \hfill \\
	As Enlight already uses a regular grid data structure to organize triangles and classify the grid's voxel as inside, outside and surface, algorithms can be used which can directly operate on these cells/voxels.
	Although the sole utilization of the cell's classification leaves a lot of information untouched, reconstructing a surface along the surface cells would be a fast way of obtaining a coarse triangle mesh of the stored workpiece.
	By using additional information inside each cell and on each grid point, well known algorithms like marching cubes can be used to retrieve a triangle mesh.
	Kobbelt \etal propose a post-processing step to the marching cubes algorithm to extract/preserve features of the represented surface by making use of triangles, points and implicit functions inside each voxel to construct a scalar distance field sampled at each grid point \cite{extended_marching_cubes}.
	DreamWorks Animation's OpenVDB library uses a similar approach for reconstructing triangle surfaces from large, sparse point clouds organized in octrees with fixed depth \cite{openvdb}.
	This algorithm is sometimes also referred to as extended marching cubes.
	The accuracy of this variant can even be further improved using dual contouring as described by Ju \etal \cite{dual_contouring}.

\end{description}



\chapter{Direct intersection}
\label{ch:direct_intersection}

basic idea
Basic idea
Why it does not work?
Run time complexity
Numerical robustness


\chapter{Method 2: Tri-dexel}
\label{ch:tri_dexel}



\section{Concept}
\label{sec:tri_dexel_concept}



\section{Implementation}
\label{sec:tri_dexel_implementation}



\subsection{Raycast}
\label{sec:tri_dexel_raycast}



\subsection{Dexel image generation}
\label{sec:tri_dexel_dexel_image_generation}



\subsection{Regularization}
\label{sec:tri_dexel_regularization}



\subsection{Triangulation}
\label{sec:tri_dexel_triangulation}



\subsection{Refinement and feature reconstruction}
\label{sec:tri_dexel_refinement}



\subsection{Subslicing *experimential*}
\label{sec:tri_dexel_subslicing}



\subsection{Parallelization}
\label{sec:tri_dexel_parallelization}



\section{Results}
\label{sec:tri_dexel_results}




\chapter{Ball pivoting algorithm}
a.	Basic idea (1)
b.	Point cloud (3)
•	Triple orthogonal raycast, (adaptive?)
c.	Ball pivoting algorithm (8)
d.	(G2S algorithm) (6)
e.	 Feature reconstruction??


\chapter{Benchmarks and comparison}

a.	Test scenes with various properties (3)
b.	Screenshots and tables with results (5)

\chapter{Conclusion}
\label{ch:conclusion}

\section{Summary}
\label{sec:summary}

% introduction
Many different representations exist for discribing solids, their volumes and surfaces.
Triangle meshes are one of the most prominent representations, as they are perfectly suited for immediate visualization using modern hardware architectures, \ie GPUs, and APIs, \eg OpenGL and DirectX.
They are also well-established for processing, storing, editing, distributing and celling geometric models.
Other representations such as depth and dexel images, spatial decomposition, functional models and CSG are mostly tailored towards specific applications and found in their repsective domains.
One example is the description of a workpiece created by subtractive manufacturing based on Boolean operations as a special kind of CSG tree.
Sometimes, these representations have to be converted into triangle meshes.

% previous work
Based on the research projects Enlight and Engrave, the Virtual Modeling Library (VML) is developed as a simulation and visualization software for subtractive manufacturing by the RISC Software GmbH.
The VML maintains a stock and several swept volume meshes in a regular grid data structure to describe a machined workpiece.
No explicit surface representation is stored by the VML.
Obtaining images of this workpiece uses a raycasting technique to sample the surface of the workpiece.

% related work and state of the art
To create a triangle mesh representation of the data model of the VML, conversion and reconstruction algorithms have been researched.
%
Directly intersecting the stock and swept volume meshes to compute their Boolean difference is a straight forward solution but calculatively expensive.
%
Using an adaption of the existing raycasting system allows the sampling of surface points.
For reconstructing triangle meshes from point clouds created this way many algorithms are available in literature and software libraries.
%
An adapted raycast does also allow the sampling of dexel and multi-dexel images from the VML.
Dexel images are rich in semantic and well-working surface reconstruction algorithms exist.
%
Based on the regular grid data structure of the VML, voxel based surface reconstruction algorithms might benefit from the existing infrastructure.

% implementations
Three approaches to create a triangle mesh from the data model of the VML have been implemented and discussed in this thesis.


% implementations: direct intersection
As first implementation, a direct intersection approach has been implemented.
The algorithm separates the distinct stock and swept volume meshes and performs a pairwise reduction until a final mesh is left.
Due to the high number of calculations involved and the limited precision of conventional computers, this algorithm is susceptible to nummerical instabilities with an increasing complexity of the scene, \eg the number of swept volumes.
Although several extensions have been proposed to mitigate these numeric issues, the algorithm was only able to reconstruct simple scenes.
The implementation provides good scalability on multiple cores but scaled directly with the number of swept volumes and their resolution, considerably increasing the runtime for complex scenes.

% implementations: tri-dexel
To provide a more robust and sampling based reconstruction, a tri-dexel based approach has been implemented.
An adapted raycast along the three axes of the coordinate system creates a tri-dexel image of the workpiece.
This image is converted into a tri-dexel grid and is regularized.
Based on this grid, boundary loops are detected in boundary cells
These loops are refined using surface normal information and triangulated.
The tri-dexel approach reconstructs all test scenes successfully and creates oriented and closed meshes.
Cell slicing is proposed as an extension to this tri-dexel reconstruction to allow adaptive resampling of features at the cost of creating T-vertices and thin holes.
The tri-dexel implementation scales well on multi-core processors and, without cell slicing, requires roughly the same runtime for all scenes.
The raycasting resolution provides a good steering parameter between quality and runtime.

% implementations: point cloud based
The same raycast used for creating a tri-dexel image may also be used to create a point cloud of the workpiece maintained by the VML.
This way, a lot of algorithms found in literature and ready-to-use libraries become available.
As examples, the BPA used internally by the VML for computing swept volumes as well as the Poisson surface reconstruction of MeshLab have been tested.
Both algorithms provide good results.
The BPA is sometimes troubled by smaller features, leading to non-manifold meshes and holes.
The Poisson surface reconstruction creates perfectly oriented, closed and manifold meshes but looses sharp features due to its robust design against noisy input.
The BPA does not scale well on multiple cores, but is heavily optimized and generally fast.
Similar to the tri-dexel variant, the raycasting resolution is used to steer quality and computational demands.


% the actual conclusion
Concluding and regarding the problem statement and goals specified in \cref{sec:problem,sec:goal}, two viable candidates have been found for extracting a surface mesh from the data model of the VML: the tri-dexel and the point cloud/BPA approach.
Regarding all the criteria given in the original problem statement, \cf \cref{sec:problem}, the tri-dexel approach slightly surpasses the results of the BPA, mostly because of its stability.
Without using cell slicing, all reconstructed meshes are perfectly oriented, closed and manifold.
Although the BPA delivers good results if the point cloud does not contain ball-sized concave features, there are still scenes, \cf the \turbine at lower resolutions, which cause the BPA to produce larger holes, non-manifold and differently oriented geometry.
Regarding visual quality, both algorithms have their troubles: the BPA omits small concave features whereas the tri-dexel generally drops small features during regularization.
The cell slicing extension of the tri-dexel approach further demonstrates a concept and prototype to greatly improve the capabilities of reconstructing features, at the consequence of T-vertices and thin holes.
Therefore, the tri-dexel algorithm will also be the focus of further research.


\section{Future work and outlook}
\label{sec:outlook}

The goal of this thesis has been mostly achieved, especially with the tri-dexel surface reconstruction implementation.
Nonetheless, the algorithm and implementation are not yet satisfying enough to be suitable for use in production code.
Although the version without cell slicing yields great mesh quality, the feature reconstructing capabilities of the cell slicing extension promise almost perfect visual outcomes.
The large amount of T-vertices and holes generated by the cell slicing version remain a problem which still has to be solved.
A post processing pass might identify these irregularities and solve them by filling the holes with triangles and splitting triangles at T-vertices.
Another solution could propagate the additional vertices created in sliced cells to the boundary loops of their neighboring cells, preventing the problem in the first place.

In its current state, the tri-dexel implementation is still notably slow when compared with fully optimized algorithms like the BPA of the VML.
There are a several potentialities for improving both runtime and memory requirements.
Apart from increasing the performance of the raycaster, the tri-dexel image intermediate representation could be eliminated and the dexels of a ray mapped directly to the tri-dexel grid, saving time and memory.
A further problem is the large amount of copied data, \eg aggregating all data of a cell from the tri-dexel grid into a new data item before processing it further.
Although these copies allow easier parallelization, the additional allocations still require memory and a not negligible amount of time.
Furthermore, the tri-dexel grid also stores full dexel segments on every edge of cells inside the workpiece.
These could be dropped without influencing the algorithm at all.

In addition to these algorithmic improvements, further optimizations of the generated triangle meshes are possible.
A desirable feature would be the ability to create meshes with adaptive resolutions, \ie triangle sizes.
Thus, the overall triangle count of the output could be reduced significantly, which is beneficial for all subsequent processing of result meshes, \eg further analysis and simulations.
Especially scenes with large flat areas, \cf the base/bottom side of almost all scenes, would greatly benefit from this adaptivity.

Finally, the algorithms based on sampling the workpiece using a raycast, \ie tri-dexel and point cloud based ones, easily allow reconstructing surfaces from various kinds of representations, as long as surface entries and exits can be sampled using rays.
This property allows the internal representation of the VML to change, enabling not only triangle mesh representations for stock and swept volumes.
Moreover, it further allows these algorithms to be used outside the VML as well, \eg to reconstruct the final mesh of a CSG tree or a functionally described shape.


% ====================================================================================================
% ===   Back matter                                                                                ===
% ====================================================================================================

\addcontentsline{toc}{chapter}{List of Figures}
\renewcommand\listfigurename{List of Figures}
\listoffigures
\clearpage

%\addcontentsline{toc}{chapter}{List of Listings}
%\renewcommand\lstlistlistingname{List of Listings} 
%\lstlistoflistings
%\clearpage

\addcontentsline{toc}{chapter}{References}
\printbibliography[title=References]

\appendix
\include{appendix}

\end{document}