
\chapter{Point cloud based methods}
\label{ch:point_cloud_based}

Reconstructing surfaces from point clouds is a lively branch in computational geometry.
As discussed when summarizing the state of the art, \cf chapter \ref{ch:state_of_the_art}, several well-working methods have been developed to convert a set of points into a polyhedral surface, \ie triangle meshes in most cases.
One possible explanation for this richness in algorithms is that point clouds are heavily used for their simplicity when digitizing real-world objects, \eg by scanning their surface with lasers or depth-sensing cameras.

Some algorithms temporarily create intermediate representations, \eg Voronoi diagrams and medial axes, like the cocone \cite{cocone, tight_cocone, robust_cocone} and crust \cite{crust, power_crust} family, or functional ones, like the MLS \cite{mls}, RBF \cite{rbf} or signed distance functions \cite{sdf_surface_reconstruction} approaches.
Thus, some of these algorithms have other uses as well beside surface reconstruction.
Other methods create surface triangles directly, mostly by growing a region on the sampled surface.
They are typically faster but more susceptible to noisy input, \eg BPA and G2S.
Nevertheless, this collection of algorithms should also be available for reconstructing surfaces from the VML's data model.

In the previous chapter \ref{ch:tri_dexel}, the tri-dexel based surface reconstruction approach relied on a raycast to sample the VML's regular grid into dexels.
This raycast may just as well be used to sample other data structures as well.
In its simplest form, by only collecting the surface hits, a point cloud is obtained.
This point cloud may then be used as input to probably all point cloud based surface reconstruction algorithms, including the ones mentioned above.


\section{Concept}
\label{sec:point_cloud_concept}

Using any kind of point cloud based algorithm for surface reconstruction from the VML's data model, boils down to two steps.
Firstly, a point cloud has to be created from the VML's data model.
This step is very similar and even simpler than the creation of the dexel image discussed in the previous chapter \ref{ch:tri_dexel}.
The same raycasting technique, with all its error correcting and robustness enhancing code, will be used to sample surface intersections on the VML's data model.
The resolution of the raycasting grid for each of the three axis-aligned casts is again supplied by the user, again providing a good parameter for steering quality and computational demands.
%
Secondly, after the point cloud has been created, a surface reconstruction algorithm is run on it.

Regarding its quality, a point cloud sampled from the VML's regular grid using a uniform, axis-aligned raycast has a few properties:
\begin{itemize}
	\item
	All surface points, excluding numeric errors, are usually perfect samples, lying directly on the workpiece's surface.
	There is neither noise on the surface nor are there irrelevant inner or outer points.
	\item
	Each surface sample provides a perfect surface normal.
	\item
	The regularity of the raycaster guarantees a minimum and maximum density of the cloud, which may be derived from the ray distances along each axis.
\end{itemize}

Despite these quality guarantees, a point cloud is still less rich in semantics when compared with a tri-dexel image.
The most prominent differences and commonalities are:
\begin{itemize}
	\item
	Point clouds only store surface information, whereas a tri-dexel image also holds volumetric information, \ie the dexel segments and point occupancy of the tri-dexel grid.
	\item
	Although a sufficient density is necessary, point clouds do not profit from the regular and uniform sampling.
	This property is fundamental for the tri-dexel approach, as it enables the construction of the regular tri-dexel grid and breaks down the problem of triangulation.
	\item
	The regularization of the tri-dexel cells provides a good way of detecting feature-rich regions which should be resampled with increased density.
	Such regions are harder to find with a point cloud, although local point distance and normal variance might give good hints.
	\item
	Although no regularization step is needed for small peculiarities of the cloud, it depends solely on the used reconstruction algorithm whether such information can be extracted or is left unused.
	Regarding the tri-dexel approach presented in chapter \ref{ch:tri_dexel}, such behavior is well-defined by either regularizing, \ie dropping, or slicing a cell with small features.
	\item
	In both cases, features which are not discovered by the raycaster are not present in the built data structures.
	Subsequent algorithms have to implement some kind of feature reconstruction.
	\item
	Point clouds are simpler data structures and easier to exchange with other 3D software.
	\item
	Point clouds have a substantially smaller memory footprint than tri-dexel images.
\end{itemize}


\section{Implementation}
\label{sec:point_cloud_implementation}

Reusing the axis-aligned raycast from section \ref{sec:tri_dexel_raycast}, all reported surface intersections are just accumulated into a set of points.
Afterwards, any kind of surface reconstruction algorithm may used to proceed.

Algorithm \ref{alg:point_cloud_based} shows the basic routine used to obtain a point cloud and call a subsequent reconstruction algorithm.
%
\begin{algorithm}
	\centering
	\begin{algorithmic}[1]
		\Function{PointCloudBased}{$\var{grid}, \var{resolution}$}
			\State $\var{res} = \Call{UniformResolution}{\var{grid}.\var{aabb}, \var{resolution}}$
			\State $\var{cloud} = \varnothing$
			\State $\Call{AxisParallelRaycast}{\var{grid}, \var{grid}.\var{aabb}, \var{res},\hfill\break
				\hspace*{\dimexpr\algorithmicindent*2}(\var{axis}, \var{x}, \var{y}, \var{v}, \var{n}) \rightarrow \var{cloud}.\var{add}((\var{v}, \var{n}))}$
			\State \Return $\Call{ReconstructFromPointCloud}{\var{cloud}}$
		\EndFunction
	\end{algorithmic}
	\caption{
		Abstract workflow of the surface reconstruction using an arbitrary point cloud reconstruction algorithm \textproc{ReconstructFromPointCloud}.
	}
	\label{alg:point_cloud_based}
\end{algorithm}
%







