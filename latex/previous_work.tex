\chapter{Previous work}
\label{ch:previous_work}

As stated in the problem statement in section \ref{sec:problem} the focus of the implementations underlying this thesis is to implement multiple strategies to extract a triangulated surface mesh from the data model used inside the VML.
To aid the reader in understanding the implementations present in this thesis, a short introduction to the VML and its data model is given.
Parts of this description have been taken from the authors bachelor thesis where the data model has been previously described with the focus on visualization \cite{bachelor}.

\section{Project history}
\label{sec:project_history}

From mid 2011 until the end of 2013 project Enlight was conducted for research by the RISC Software GmbH in Hagenberg im M\"uhlkreis, Austria.
Enlight's goals were to develop a faster, scalable and numerically stable method for modeling and visualizing subtractive manufacturing.
Enlight used a regular grid data structure to store a stock solid and add precomputed swept volumes.
A triangle elimination strategy was employed to keep the total number of triangles held by the grid within manageable bounds.
For visualization a customized ray casting approach was developed \cite{enlight} and accelerated using GPUs and many-core architectures.

From the beginning to the end of 2014, the follow-up research project Engrave focused on solving swept volume computation for arbitrary cutter geometries and tool paths.
Engrave basically allowed dynamic swept volume computation from a set of cutter solids and transformation lists.
Swept volume computation was done by extruding a point cloud along the tool path and then reconstruction a closed triangle mesh from it using a parallel and highly optimized variant of the ball pivoting algorithm 
\cite{engrave}.
The computed swept volumes where directly imported into Enlight's data model.

Both projects, Enlight and Engrave, were co-funded by the European Union as well as Land Ober\"osterreich within the political program Regio 13, which aimed to sustainably improve the contestability of regional companies, economic growth and employment inside of Upper Austria.
%
With the beginning of 2015 the prototype developed during Enlight and Engrave was rebranded to Virtual Machining Library (VML) and is currently further developed as a commercial product.

\chapter{VML data model}
\label{sec:vml_data_model}


%TODO


The models in Enlight are defined by a set of triangle meshes which are combined using set operations (particularly subtraction).
Applications of such a system are primarily found in simulations of material removal processes, a core discipline of machining, also referred to as subtractive manufacturing.
A vital requirement of such systems is to support a large number of fine triangulated meshes (up to hundreds of millions of triangles) to deliver simulations of appropriate quality.
Therefore, well-conceived space partitioning and triangle elimination strategies have been developed to deal with this amount of input.
As a result the simulated model is described implicitly via subtraction of only subsets of the input triangles (volumes are no longer closed/water-tight).
For visualization, this implicit model is sampled using an adapted ray casting approach \cite{enlight}.
By applying a small set of rules, the surface can be recovered to retrieve an image of the model.
Enlight's data model is described in detail in the author's bachelor thesis \cite{bachelor}.

Simulation and visualization of material removal processes such as in Enlight are vital for verifying the correctness of a machining process.
However, there exist numerous scenarios where an explicit surface model of the final product is required.
Many of these situations occur in CAE such as stress, thermal and safety analyses as well as structural optimizations using finite element methods.
Due to the importance for CAE for assuring quality in modern production processes it is highly desired to be able to retrieve an explicit surface mesh of the implicitly described model of Enlight.



more or less practical background

Enlight, RISC Software GmbH, practical bachelor thesis, internship, Enlight/Engrave papers
