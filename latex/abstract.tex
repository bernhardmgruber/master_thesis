\chapter*{Kurzfassung}


\pagebreak

\chapter*{Abstract}

Using the Virtual Modeling Library (VML) developed at the RISC Software GmbH as an example, this thesis discusses surface reconstruction algorithms from data models used to simulate subtractive manufacturing.
Specifically, to describe a machined workpiece created during this process, the VML stores a stock and a series of swept volume triangle meshes, similar to a CSG tree.
Via a custom raycasting process, surface points of the workpiece are sampled to produce visualizations of the scene.

Sometimes, an explicit surface mesh representation of the workpiece is desired.
To reconstruct such a mesh, three approaches are presented.

The first one relies on calculating the boolean subtraction of the maintained stock and swept volume by direct mesh intersection.
Although producing exact results for simple scenes, this method fails for complex scenes due to numerical issues.

The second approach samples a Tri-dexel image using an adapted raycast and reconstructs a triangle mesh from the tri-dexel representation.
With overall good robustness, an acceptable visual outcome and excellent mesh qualities, this method works well for all tested scenarios.
Additionally, a novel extension further improves the reconstruction of features at the cost of creating thin holes in the surface.

Finally, a raycast may also be used to sample a point cloud from the workpiece.
By the example of the Ball Pivoting Algorithm (BPA) of the VML and a Poisson surface reconstruction implementation of MeshLab, the use of point cloud based surface reconstruction algorithms is discussed as well.
