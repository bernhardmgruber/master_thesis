\chapter*{Kurzfassung}

\begin{otherlanguage}{ngerman}
	Am Beispiel der Virtual Modeling Library (VML), die an der RISC Software GmbH entwickelt wird, diskutiert diese Arbeit Algorithmen zur Oberflächenrekonstruktion aus Datenmodellen zur Simulation zerspanender Bearbeitungsprocesse.
	Um ein Werkstück einer derartigen Bearbeitung zu beschreiben, verwaltet die VML Dreiecksmodelle des Ausgangs- und einer Serie an Abzugsvolumen, ähnlich einem CSG Baum.
	Unter Verwendung eines angepassen Raycastingprozesses werden Oberflächenpunkte des Werkstücks abgetastet um Visualisierungen der Szene zu erzeugen.
	
	Für bestimmte Anwendungsfälle ist jedoch eine explizite Repräsentation des Werkstücks als Oberflächennetz gewünscht.
	Um ein derartiges Netz zu erzeugen werden drei Methoden vorgestellt.
	
	Die erste errechnet die Boolschen Subtraktionen der Ausgangs- und Abzugsvolumen durch direkte Verschneidung der jeweiligen Dreiecksmodelle.
	Obwohl für einfache Szenen exakte Resultate berechnet werden können, scheitert diese Methode an komplexen Szenen aufgrund numerischer Probleme.
	
	Die zweite Methode tastet unter Verwendung eines adaptierten Raycastings ein Tri-Dexel-Bild ab und rekonstruiert daraus ein Dreiecksnetz.
	Mit durchwegs guter Robustheit, einem visuell vertretbaren Resultat und exzellenter Qualität des Netzes, funktioniert diese Methode gut in allen getesteten Szenarios.
	Zusätzlich verbessert eine neuartige Erweiterung des Tri-Dexel Algorithmus die Rekonstruktion von Details auf Kosten kleiner Löcher in der erzeugten Oberfläche.
	
	Als dritte Methode kann ein Raycast auch zur Abtastung einer Punktwolke der Werkstücksoberfläche herangezogen werden.
	Am Beispiel des Ball Pivoting Algorithm der VML und der Implementierung einer Poisson Oberflächenrekonstruktion in MeshLab werden noch die Verwendung von punktwolkenbasierten Algorithmen diskutiert.
\end{otherlanguage}

\pagebreak

\chapter*{Abstract}

Using the Virtual Modeling Library (VML) developed at the RISC Software GmbH as an example, this thesis discusses surface reconstruction algorithms from data models used to simulate subtractive manufacturing.
To describe a machined workpiece created during this process, the VML stores a stock and a series of swept volume triangle meshes, similar to a CSG tree.
Via a custom raycasting process, surface points of the workpiece are sampled to produce visualizations of the scene.

Sometimes, an explicit surface mesh representation of the workpiece is desired.
To reconstruct such a mesh, three approaches are presented.

The first one relies on calculating the Boolean subtraction of the maintained stock and swept volumes by direct mesh intersection.
Although producing exact results for simple scenes, this method fails for complex scenes due to numerical issues.

The second approach samples a Tri-dexel image using an adapted raycast and reconstructs a triangle mesh from the tri-dexel representation.
With overall good robustness, an acceptable visual outcome and excellent mesh qualities, this method works well for all tested scenarios.
Additionally, a novel extension to the tri-dexel algorithm further improves the reconstruction of features at the cost of thin holes in the created surface.

As a third method, a raycast may also be used to sample a point cloud of the surface of the workpiece.
By the example of the Ball Pivoting Algorithm of the VML and a Poisson surface reconstruction implementation of MeshLab, the use of point cloud based surface reconstruction algorithms is discussed as well.
