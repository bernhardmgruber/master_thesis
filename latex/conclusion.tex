\chapter{Conclusion}
\label{ch:conclusion}

\section{Summary}
\label{sec:summary}

% introduction
Many different representations exist for discribing solids, their volumes and surfaces.
Triangle meshes are one of the most prominent representations, as they are perfectly suited for immediate visualization using modern hardware architectures, \ie GPUs, and APIs, \eg OpenGL and DirectX.
They are also well-established for processing, storing, editing, distributing and celling geometric models.
Other representations such as depth and dexel images, spatial decomposition, functional models and CSG are mostly tailored towards specific applications and found in their repsective domains.
One example is the description of a workpiece created by subtractive manufacturing based on Boolean operations as a special kind of CSG tree.
Sometimes, these representations have to be converted into triangle meshes.

% previous work
Based on the research projects Enlight and Engrave, the Virtual Modeling Library (VML) is developed as a simulation and visualization software for subtractive manufacturing by the RISC Software GmbH.
The VML maintains a stock and several swept volume meshes in a regular grid data structure to describe a machined workpiece.
No explicit surface representation is stored by the VML.
Obtaining images of this workpiece uses a raycasting technique to sample the surface of the workpiece.

% related work and state of the art
To create a triangle mesh representation of the data model of the VML, conversion and reconstruction algorithms have been researched.
%
Directly intersecting the stock and swept volume meshes to compute their Boolean difference is a straight forward solution but calculatively expensive.
%
Using an adaption of the existing raycasting system allows the sampling of surface points.
For reconstructing triangle meshes from point clouds created this way many algorithms are available in literature and software libraries.
%
An adapted raycast does also allow the sampling of dexel and multi-dexel images from the VML.
Dexel images are rich in semantic and well-working surface reconstruction algorithms exist.
%
Based on the regular grid data structure of the VML, voxel based surface reconstruction algorithms might benefit from the existing infrastructure.

% implementations
Three approaches to create a triangle mesh from the data model of the VML have been implemented and discussed in this thesis.


% implementations: direct intersection
As first implementation, a direct intersection approach has been implemented.
The algorithm separates the distinct stock and swept volume meshes and performs a pairwise reduction until a final mesh is left.
Due to the high number of calculations involved and the limited precision of conventional computers, this algorithm is susceptible to nummerical instabilities with an increasing complexity of the scene, \eg the number of swept volumes.
Although several extensions have been proposed to mitigate these numeric issues, the algorithm was only able to reconstruct simple scenes.
The implementation provides good scalability on multiple cores but scaled directly with the number of swept volumes and their resolution, considerably increasing the runtime for complex scenes.

% implementations: tri-dexel
To provide a more robust and sampling based reconstruction, a tri-dexel based approach has been implemented.
An adapted raycast along the three axes of the coordinate system creates a tri-dexel image of the workpiece.
This image is converted into a tri-dexel grid and is regularized.
Based on this grid, boundary loops are detected in boundary cells
These loops are refined using surface normal information and triangulated.
The tri-dexel approach reconstructs all test scenes successfully and creates oriented and closed meshes.
Cell slicing is proposed as an extension to this tri-dexel reconstruction to allow adaptive resampling of features at the cost of creating T-vertices and thin holes.
The tri-dexel implementation scales well on multi-core processors and, without cell slicing, requires roughly the same runtime for all scenes.
The raycasting resolution provides a good steering parameter between quality and runtime.

% implementations: point cloud based
The same raycast used for creating a tri-dexel image may also be used to create a point cloud of the workpiece maintained by the VML.
This way, a lot of algorithms found in literature and ready-to-use libraries become available.
As examples, the BPA used internally by the VML for computing swept volumes as well as the Poisson surface reconstruction of MeshLab have been tested.
Both algorithms provide good results.
The BPA is sometimes troubled by smaller features, leading to non-manifold meshes and holes.
The Poisson surface reconstruction creates perfectly oriented, closed and manifold meshes but looses sharp features due to its robust design against noisy input.
The BPA does not scale well on multiple cores, but is heavily optimized and generally fast.
Similar to the tri-dexel variant, the raycasting resolution is used to steer quality and computational demands.


% the actual conclusion
Concluding and regarding the problem statement and goals specified in \cref{sec:problem,sec:goal}, two viable candidates have been found for extracting a surface mesh from the data model of the VML: the tri-dexel and the point cloud/BPA approach.
Regarding all the criteria given in the original problem statement, \cf \cref{sec:problem}, the tri-dexel approach slightly surpasses the results of the BPA, mostly because of its stability.
Without using cell slicing, all reconstructed meshes are perfectly oriented, closed and manifold.
Although the BPA delivers good results if the point cloud does not contain ball-sized concave features, there are still scenes, \cf the \turbine at lower resolutions, which cause the BPA to produce larger holes, non-manifold and differently oriented geometry.
Regarding visual quality, both algorithms have their troubles: the BPA omits small concave features whereas the tri-dexel generally drops small features during regularization.
The cell slicing extension of the tri-dexel approach further demonstrates a concept and prototype to greatly improve the capabilities of reconstructing features, at the consequence of T-vertices and thin holes.
Therefore, the tri-dexel algorithm will also be the focus of further research.


\section{Future work and outlook}
\label{sec:outlook}

The goal of this thesis has been mostly achieved, especially with the tri-dexel surface reconstruction implementation.
Nonetheless, the algorithm and implementation are not yet satisfying enough to be suitable for use in production code.
Although the version without cell slicing yields great mesh quality, the feature reconstructing capabilities of the cell slicing extension promise almost perfect visual outcomes.
The large amount of T-vertices and holes generated by the cell slicing version remain a problem which still has to be solved.
A post processing pass might identify these irregularities and solve them by filling the holes with triangles and splitting triangles at T-vertices.
Another solution could propagate the additional vertices created in sliced cells to the boundary loops of their neighboring cells, preventing the problem in the first place.

In its current state, the tri-dexel implementation is still notably slow when compared with fully optimized algorithms like the BPA of the VML.
There are a several potentialities for improving both runtime and memory requirements.
Apart from increasing the performance of the raycaster, the tri-dexel image intermediate representation could be eliminated and the dexels of a ray mapped directly to the tri-dexel grid, saving time and memory.
A further problem is the large amount of copied data, \eg aggregating all data of a cell from the tri-dexel grid into a new data item before processing it further.
Although these copies allow easier parallelization, the additional allocations still require memory and a not negligible amount of time.
Furthermore, the tri-dexel grid also stores full dexel segments on every edge of cells inside the workpiece.
These could be dropped without influencing the algorithm at all.

In addition to these algorithmic improvements, further optimizations of the generated triangle meshes are possible.
A desirable feature would be the ability to create meshes with adaptive resolutions, \ie triangle sizes.
Thus, the overall triangle count of the output could be reduced significantly, which is beneficial for all subsequent processing of result meshes, \eg further analysis and simulations.
Especially scenes with large flat areas, \cf the base/bottom side of almost all scenes, would greatly benefit from this adaptivity.

Finally, the algorithms based on sampling the workpiece using a raycast, \ie tri-dexel and point cloud based ones, easily allow reconstructing surfaces from various kinds of representations, as long as surface entries and exits can be sampled using rays.
This property allows the internal representation of the VML to change, enabling not only triangle mesh representations for stock and swept volumes.
Moreover, it further allows these algorithms to be used outside the VML as well, \eg to reconstruct the final mesh of a CSG tree or a functionally described shape.
